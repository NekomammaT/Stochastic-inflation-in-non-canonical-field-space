\documentclass[a4paper,11pt]{article}
\pdfoutput=1
\usepackage{jcappub}
\usepackage{graphicx}
\usepackage[caption=false]{subfig}
\usepackage{mathrsfs}
\usepackage{amsmath,amssymb}
\usepackage{bm}
\usepackage{braket}
\usepackage{listings}
\usepackage{cases}
\usepackage{comment}
\usepackage{soul}
\usepackage{cancel}
\usepackage{cases}
\usepackage[utf8]{inputenc}
\usepackage{url}
\usepackage{longtable}
\usepackage[normalem]{ulem}
%\usepackage{hyperref}

%\usepackage{mathpazo}
%\usepackage[no-math]{fontspec}
%\setmainfont{Palatino}
%\setsansfont{Optima}

\newcommand{\dif}[2]{\frac{\mathrm{d} #1}{\mathrm{d} #2}}
\newcommand{\pdif}[2]{\frac{\partial #1}{\partial #2}}
\newcommand{\var}[2]{\frac{\delta #1}{\delta #2}}
\newcommand{\dd}{\mathrm{d}}
\newcommand{\DD}{\mathscr{D}}
\newcommand{\ee}{\mathrm{e}}
\newcommand{\diag}{\mathrm{diag}}
\newcommand{\sgn}{\mathrm{sgn}}
\newcommand{\Mpl}{M_\text{Pl}}
\newcommand{\ns}{n_{{}_\mathrm{S}}}
\newcommand{\cs}{c_{{}_\mathrm{S}}}
\newcommand{\IR}{\text{IR}}
\newcommand{\UV}{\text{UV}}
\renewcommand{\Re}{\mathrm{Re}}
\renewcommand{\Im}{\mathrm{Im}}
\newcommand{\dk}{\frac{\dd^3k}{(2\pi)^3}}
\newcommand{\bbalpha}{{\alpha\!\!\!\alpha}}
\newcommand{\dps}{\displaystyle}
\newcommand{\SIA}{S_\text{IA}}
\newcommand{\eff}{\text{eff}}
\newcommand{\kdx}{\mathbf{k}\cdot\mathbf{x}}
\newcommand{\Pbraket}[1]{\left\{ #1 \right\}}

\newcommand{\calD}{\mathcal{D}}
\newcommand{\scrD}{\mathscr{D}}
\newcommand{\calg}{\mathcal{g}}
\newcommand{\calH}{\mathcal{H}}
\newcommand{\scrH}{\mathscr{H}}
\newcommand{\calJ}{\mathcal{J}}
\newcommand{\scrJ}{\mathscr{J}}
\newcommand{\calL}{\mathcal{L}}
\newcommand{\scrL}{\mathscr{L}}
\newcommand{\calN}{\mathcal{N}}
\newcommand{\calO}{\mathcal{O}}
\newcommand{\scrO}{\mathscr{O}}
\newcommand{\calP}{\mathcal{P}}
\newcommand{\calR}{\mathcal{R}}

\newcommand{\bae}[1]{\begin{align} #1 \end{align}}
\newcommand{\bce}[1]{\begin{cases} #1 \end{cases}}
\newcommand{\bfe}[4]{
\begin{figure} 
	\centering
	\includegraphics[#1]{#2}
	\caption{#3}
	\label{#4}
\end{figure}}
\newcommand{\bpme}[1]{\begin{pmatrix} #1 \end{pmatrix}}

\newcommand{\Red}[1]{\textcolor{red}{\sffamily #1}}
\newcommand{\Mag}[1]{\textcolor{magenta}{\sffamily #1}}
\newcommand{\Blue}[1]{\textcolor{blue}{\sffamily #1}}
\newcommand{\mathblue}[1]{\textcolor{blue}{#1}}




\title{Stochastic inflation in non-canonical field space}

\author[a]{Lucas Pinol,}
\author[a]{Sébastien Renaux-Petel,}
\author[a,b]{and Yuichiro Tada}

\affiliation[a]{Institut d'Astrophysique de Paris, GReCO, UMR 7095 du CNRS et de Sorbonne Universit\'e, 98bis boulevard Arago, 75014 Paris, France}
\affiliation[b]{Department of Physics, Nagoya University, Nagoya 464-8602, Japan}

\emailAdd{pinol@iap.fr}
\emailAdd{renaux@iap.fr}
\emailAdd{tada.yuichiro@e.mbox.nagoya-u.ac.jp}

\abstract{
The stochastic inflation is formulated in a generalized non-flat field space.
Including metric perturbations, we derive the Langevin equation in a covariant language with use of the effective Hamiltonian action approach
as well as reviewing the heuristic equation of motion method.
It is clarified that the Friedmann equation still holds in each Hubble patch without any stochastic noise
since the correction terms in the effective action do not explicitly depend on the lapse function.
We also mention the uncertainty on the interpretation of noise integral.
}

\keywords{}
\arxivnumber{}


\begin{document}
\maketitle
\flushbottom
%\tableofcontents


\section{Introduction}\label{sec: introduction}

Recent cosmic microwave background (CMB) data from the Planck and BICEP2/Keck collaborations
severely constrain the physics of the early universe as the spectral index of primordial curvature perturbations $\ns=0.968\pm0.006$ ($95\%$ C.L.)
and the tensor-to-scalar ratio $r<0.12$ ($95\%$ C.L.)~\cite{Ade:2015lrj,Ade:2015tva}.
Inflation~\cite{Starobinsky:1980te,Sato:1980yn,Guth:1980zm,Linde:1981mu,Albrecht:1982wi,Linde:1983gd} 
is one of the most plausible scenario consistent with these observations,
which naturally yields almost scale-invariant curvature perturbations
from the quantum zero point fluctuations as well as solves the crucial problems of 
the big-bang theory~\cite{Mukhanov:1981xt,Hawking:1982cz,Starobinsky:1982ee,Guth:1982ec,Bardeen:1983qw}.
Despite of its great success, the concrete mechanism of inflation has not been clarified yet.
In such a situation, represented by e.g. $\alpha$-attractor~\cite{}, the geometrical destabilization of inflation~\cite{Renaux-Petel:2015mga,Renaux-Petel:2017dia,
Garcia-Saenz:2018ifx}, hyperinflation~\cite{Brown:2017osf,Mizuno:2017idt}, and so on, the possibility of non-canonical kinetic term has attracted more attentions
other than the tuning of the potential term. Indeed the field space metric encoded by the kinetic term is not necessarily flat in multi-inflaton cases in general.

One of the key feature distinguishing inflationary models is scalar fluctuations. Pioneered by Starobinsky, the so-called stochastic formalism~\cite{Starobinsky:1986fx,
Nambu:1987ef,Nambu:1988je,Kandrup:1988sc,Nakao:1988yi,Nambu:1989uf,Mollerach:1990zf,
Linde:1993xx,Starobinsky:1994bd,Rigopoulos:2004gr,Rigopoulos:2005xx,Finelli:2008zg,Finelli:2010sh} is known as a powerful tool to investigate 
the diffusion dynamics of scalar fields due to the quantum zero-point fluctuation. In this formalism, the superhorizon coarse-grained fields are
considered as background fields and then the subhorizon modes exiting the horizon are naturally interpreted as random noise, reproducing the scalar fluctuations.

Though it is derived in a fully slow-rolled assumption originally, the phase space expression from the Hamilton-Jacobi equation has been addressed in 
Refs.~\cite{Tolley:2008na,Grain:2017dqa,Ezquiaga:2018gbw}, enabling us to go beyond the slow-roll limit and include metric perturbations.
In \cite{Tolley:2008na}, a non-canonical field-space metric is also considered. On the other hand, following the path integral formulation of the quantum 
Brownian motion~\cite{Schwinger:1960qe,Feynman:1963fq}, several authors work on the effective action formalism for 
the stochastic inflation~\cite{Morikawa:1989xz,Matarrese:2003ye,Levasseur:2013ffa,Moss:2016uix,Tokuda:2017fdh}, 
which can be naturally extended to loop calculations beyond leading order, though without metric perturbations.
In this paper, we completely formulate the phase space stochastic formalism in the non-canonical field space with use of the effective Hamiltonian action
including metric perturbations. Thanks to the metric perturbation, we also clarify that the Friedmann constraint does hold in each Hubble patch
even in the stochastic formalism.

The rest of this paper is organized as follows. In Sec.~\ref{sec: heuristic approach}, we review the stochastic formalism, implying 
the conclusive Langevin equation in a heuristic equation of motion approach. 
The derivation of this equation in terms of the effective Hamiltonian action 
is shown in Sec.~\ref{sec: effective action}. We also mention the practical implementation of this formalism in Sec.~\ref{sec: complementaries}.
Sec.~\ref{sec: conclusions} is devoted to conclusions.






\section{Stochastic formalism: heuristic approach}\label{sec: heuristic approach}

The quantum nature of the scalar fields is responsible for fluctuations of their amplitudes around their classical values. 
In standard inflation, each field is decomposed into a homogeneous background and a small quantum perturbation. 
This setup can be justified in some cases, for example when the fields are heavy and the classical trajectory dominates the dynamics. 
One should however be aware that this is only a good approximation. 
Indeed any two points beyond the horizon scale are decoupled by causality and therefore there is no reason that the universe is dominated by
homogeneous fields in general. For light fields, inhomogeneities can develop on superhorizon scale and determines their dynamics.
Based on these facts, the stochastic formalism~\cite{Starobinsky:1986fx,Nambu:1987ef,Nambu:1988je,Kandrup:1988sc,Nakao:1988yi,
Nambu:1989uf,Mollerach:1990zf,Linde:1993xx,Starobinsky:1994bd,Rigopoulos:2004gr,Rigopoulos:2005xx,Finelli:2008zg,Finelli:2010sh} 
directly treats the coarse-grained fields on
the horizon scale as backgrounds.
The key feature of the dynamics of the coarse-grained modes is that
they seem to receive random noise because the subhorizon modes which is quantum mechanically uncertain exit the horizon and 
join to the coarse-grained fields every moment. 
These fluctuations, in the stochastic formalism, may determine the dynamics of light fields and also will cause primordial curvature perturbations
after inflation~\cite{Fujita:2013cna,Fujita:2014tja,Vennin:2015hra}.

In this section, we recall the principles of the stochastic formalism expressed in a Hamiltonian language~\cite{Tolley:2008na,Grain:2017dqa,Ezquiaga:2018gbw}, 
but applying it to the most general case of a non-canonical field space. 
This enables to identify clearly the true degrees of freedom of the theory in phase space beyond the slow-roll limit, 
and express the equations in a covariant way. 
We also include the back-reaction of the fields on space-time by allowing perturbations of the metric. 
The general action of several scalar fields minimally coupled to gravity is
\bae{\label{eq: original action}
	S=\int\dd^4x\scrL=\int\dd^4x\sqrt{-g}\left[\frac{1}{2}\Mpl^2\calR-\frac{1}{2}g^{\mu\nu}G_{IJ}(\phi)\partial_\mu\phi^I\partial_\nu\phi^J-V(\phi)\right],
}
where $\calR$ is the Ricci scalar for the spacetime metric $g_{\mu\nu}$. The field space metric $G_{IJ}$ is not necessarily flat in general.
For the spacetime metric, we adopt the ADM formalism
\bae{\label{eq: ADM}
	\dd s^2=-N^2\dd t^2+\gamma_{ij}(\dd x^i+\beta^i\dd t)(\dd x^j+\beta^j\dd t).
}
$N$ is the lapse function, $\beta^i$ is the shift vector, and $\gamma_{ij}$ is the spatial symmetric metric.
%One can see that the lapse and shift vector appear in the action as simple Lagrange multipliers, that is, the action does not include any
%time derivative of them. Therefore their four d.o.f. are killed by the corresponding constraints. Other four d.o.f. in the spatial metric can be fixed
%by choosing a gauge. 
We adopt the spatially flat gauge $\gamma_{ij}(t,\mathbf{x})=a^2(t)(\delta_{ij}+h_{ij}(t,\mathbf{x}))$ 
where $h_{ij}$ is the transverse traceless tensor $\partial_ih_{ij}=h_{ii}=0$.\footnote{In general, the scale factor will be inhomogeneous 
through the Friedmann equation
since the background coarse-grained fields are fluctuated. That is, the spatially flat gauge condition is violated by noise terms, strictly speaking.
By taking the scale factor itself (or equivalently the e-folding number which is the logarithm of the scale factor)
as a time variable, the scale factor can be kept spatially homogeneous on a equal time slice.} 
Throughout this paper, we will neglect the tensor perturbations. Indeed at least in the subhorizon 
phase, the tensor modes are decoupled from the scalar perturbations since we only consider the linear perturbation theory. 
The Ricci scalar for three-dimensional space-like hypersurfaces thus vanishes and the action reads~\cite{Salopek:1990jq}
\bae{
	S=\int\dd^4xa^3N\left[\frac{\Mpl^2}{2}\left(K_{ij}K^{ij}-K^2\right)+\frac{1}{2N^2}G_{IJ}v^Iv^J-\frac{1}{2a^2}G_{IJ}\partial_i\phi^I\partial_i\phi^J-V\right],
}
where
\bae{
	K_{ij}=\frac{1}{2N}(\beta_{i,j}+\beta_{j,i}-2a^2\calH\delta_{ij}), \quad\quad K=K^i_i, \quad\quad v^I=\dot{\phi}^I-\beta^i\partial_i\phi^I.
}
The spatial indices $i$ and $j$ are raised and lowered by the spatial metric $\gamma_{ij}=a^2\delta_{ij}$ and $\gamma^{ij}=a^{-2}\delta_{ij}$.
Commas and dots represent spatial and time derivatives and the curly $\calH=\dot{a}/a$ stands for the generalized Hubble parameter
for arbitrary $N$, while the normal $H$ is left for the standard Hubble parameter with $N=1$.

Let us now define the canonically conjugate variables of our theory as
\bae{\label{eq: Legendre trs}
	\pi^{ij}&=\var{S}{\dot{\gamma}_{ij}}=a^3(K\gamma^{ij}-K^{ij}), \quad\quad
	\pi_I=\var{S}{\dot{\phi}^I}=\frac{a^3}{N}G_{IJ}(\dot{\phi}^I-\beta^i\partial_i\phi^I), \nonumber \\
	\pi_N&=\var{S}{\dot{N}}=0, \quad\quad
	\pi_{\beta^i}=\var{S}{\dot{\beta}^i}=0.
}
The Poisson brackets of these variables are defined by
\bae{
	\Pbraket{\gamma_{ij}(\mathbf{x}),\pi^{kl}(\mathbf{y})}
	&=\frac{1}{2}(\delta_i^k\delta_j^l+\delta_i^l\delta_j^k)\delta^{(3)}(\mathbf{x}-\mathbf{y}), \quad\quad
	\Pbraket{\phi^I(\mathbf{x}),\pi_J(\mathbf{y})}=\delta^I_J\delta^{(3)}(\mathbf{x}-\mathbf{y}), \nonumber \\
	\Pbraket{N(\mathbf{x}),\pi_N(\mathbf{y})}&=\delta^{(3)}(\mathbf{x}-\mathbf{y}), \quad\quad
	\Pbraket{\beta^i(\mathbf{x}),\pi_{\beta^j}(\mathbf{y})}=\delta^i_j\delta^{(3)}(\mathbf{x}-\mathbf{y}).
}
It is worth noting that the conjugate momenta of $N$ and $\beta^i$ vanish and therefore they are non-dynamical modes.
It is the strong point of the ADM formalism that the non-dynamical modes appear clearly in the action in this way.
The corresponding Hamiltonian equation will give the constraint equations as we will see.

With these variables, the Hamiltonian density given by
\bae{
	\scrH=\pi^{ij}\dot{\gamma}_{ij}+\pi_I\dot{\phi}^I-\calL,
}
can be rewritten in a convenient form as
\bae{\label{eq: Hamiltonian}
	\scrH=N(C^G+C^\phi)+\beta^i(C_i^G+C_i^\phi),
}
where
\bae{
	C^G&=\frac{2\Mpl^2}{a^3}\left[\pi_{ij}\pi^{ij}-\frac{1}{2}(\pi^i_i)^2\right], \quad\quad
	C^G_i=-2\Mpl^2\gamma_{ij}\partial_k\pi^{jk}, \nonumber \\
	C^\phi&=\frac{G^{IJ}}{2a^3}\pi_I\pi_J+\frac{G_{IJ}a^3}{2}\gamma^{ij}\partial_i\phi^I\partial_j\phi^J+a^3V, \quad\quad
	C^\phi_i=\pi_I\partial_i\phi^I.
}
The superscripts $G$ and $\phi$ represent the gravitational and inflatons sectors respectively.
Any function $F$ of the canonical variables evolves according to the Hamiltonian evolution $\dot{F}=\Pbraket{F,\int\dd^3x\scrH}$.
Applying it to the canonical variables themselves, one obtains the equations of motion (EoM) of the system, as well as the constraint equations as
\bae{
	\dot{\phi}^I&=\frac{N}{a^3}G^{IJ}\pi_J+\beta^i\partial_i\phi^I, \label{eq: phi EoM}\\
	\dot{\pi}_I&=-a^3NV_I+\partial_i\left(aNG_{IJ}\partial_i\phi^J\right)-\frac{aN}{2}G_{JK,I}\partial_i\phi^J\partial_i\phi^K
	-\frac{N}{2a^3}G^{JK}{}_{,I}\pi_J\pi_K+\partial_i(\beta^i\pi_I), \label{eq: pi EoM} \\
	0=\dot{\pi}_N&=-(C^G+C^\phi), \label{eq: energy const} \\
	0=\dot{\pi}_{\beta^i}&=-(C^G_i+C^\phi_i). \label{eq: momentum const}
}
Here $V_I=\partial V/\partial\phi^I$, $G_{IJ,K}=\partial G_{IJ}/\partial\phi^K$, and so on.
Eqs.~(\ref{eq: energy const}) and (\ref{eq: momentum const}) can be seen as the energy and momentum constraints.
Solving them, one can explicitly express the metric perturbations in terms of the inflaton fields consistently with the fact that $N$ and $\beta^i$
are non-dynamical modes.

Taking the homogeneous limit of these equations, one obtains the classical EoM and the Friedmann equation
\bae{\label{eq: classical EoM}
	\dot{\phi}^I_0=\frac{N_0}{a^3}G^{IJ}\pi_{0J}, \quad\quad D_t\pi_{0I}=-a^3N_0V_I, \quad\quad 
	\frac{3\Mpl^2\calH^2}{N_0^2}=\frac{1}{2a^6}G^{IJ}\pi_{0I}\pi_{0J}+V,
}
while the momentum constraint is trivial. The subscript $0$ represents the homogeneous mode of variables.
$D_t$ is a covariant derivative
\bae{
	D_t\pi_{0I}=\dot{\pi}_{0I}-\Gamma_{IJ}^K\dot{\phi}_0^J\pi_{0K},
}
with the affine connection $\Gamma_{IJ}^K$ for the field space metric $G_{IJ}$.
On the other hand, in the stochastic framework, all the fields are divided into classical superhorizon IR modes and 
quantum subhorizon UV perturbations. That is, the fields and the metric components are decomposed into IR and UV parts as
\bae{\label{eq: IRUV decomposition}
	\phi^I=\varphi^I+Q^I, \quad\quad \pi_I=\varpi_I+P_I, \quad\quad N=N_\IR+\alpha, \quad\quad \beta^i=0+a^{-2}\partial_i\psi,
}
where $Q^I$, $P_I$, $\alpha$, and $\psi$ are their UV perturbations.
Each UV component of a quantity $X$ is defined by
\bae{
	X_\UV(x)=\int\dk\ee^{i\kdx}W(k/k_\sigma(t))X(t,\mathbf{k}),
}
with some window function $W$ taking only short-wavelength modes $k>k_\sigma(t)$.
The simple step function $\theta(k-\sigma aH)$ is often used as a window function as we will do so.
The time dependent cutoff scale $k_\sigma^{-1}=1/\sigma aH$ with a small model parameter $\sigma\ll1$ 
is chosen to be larger enough than the Hubble scale
$1/aH$ to allow the gradient expansion (namely the $\sigma$-expansion). 
In that way the remaining IR modes $\varphi^I$ and $\varpi_I$ with $k<k_\sigma(t)$ are well above the comoving Hubble radius
and can be considered as classicalized.
$N_\IR$ and $\beta^i_\IR$ are pure gauge choices in the long-wavelength limit
and one can fix $\beta^i_\IR=0$, while we kept $N_\IR$ for arbitrary choice of the time variable. 
Also since we will only consider the linear theory for UV modes, the vector component in $\beta^i$ is decoupled and 
only the scalar component $\psi$ will be taken into account.

While we consider the linear theory for UV modes, we do not have to use perturbative expansions in IR modes with use of the gradient expansion instead.
That is, substituting the decomposition~(\ref{eq: IRUV decomposition}) into the original EoM~(\ref{eq: phi EoM}) and (\ref{eq: pi EoM}) 
and keeping the linear terms in UV modes and the leading terms in $\sigma$, 
one can obtain the non-linear EoM for IR modes with the corrections by UV modes as
\bae{\label{eq: decomposed EoM}
	\bce{
		\dps
		\dot{\varphi}^I=\frac{N_\IR}{a^3}G^{IJ}\varpi_J-\int\dk\ee^{i\kdx}\left[\dot{W}(k/k_\sigma)\phi^I(t,\mathbf{k})
		-W(k/k_\sigma)E_Q^I(t,\mathbf{k})\right], \\[10pt]
		\dps
		D_t\varpi_I=-a^3N_\IR V_I-\int\dk\ee^{i\kdx}\left[\dot{W}(k/k_\sigma)\tilde{\pi}_I(t,\mathbf{k})
		-W(k/k_\sigma)E_{PI}(t,\mathbf{k})\right].
	}
}
where 
\bae{
	\tilde{\pi}_I(t,\mathbf{k})=\pi_I(t,\mathbf{k})-\Gamma_{IJ}^K\varpi_J\phi^K(t,\mathbf{k}), \quad\quad
	D_t\varpi_I=\dot{\varpi}_I-\Gamma_{IJ}^K\dot{\varphi}^J\varpi_K.
}
$E_Q^I$ and $E_{PI}$ are the linearized EoM for the UV modes $Q$ and $P$ in Fourier space, which should vanish in the linear perturbation theory.
Their explicit forms are shown in Sec.~\ref{sec: UV EoM}. 
Therefore we assume $E_Q^I=0=E_{PI}$,\footnote{One can check the covariance of Eq.~(\ref{eq: decomposed EoM})
with use of these UV EoM.} 
but there are additional terms from the time-derivative of the window function other than them
since the coarse-graining scale $k_\sigma=\sigma aH$ is now time-dependent.
They represent the effect that the UV modes cross the coarse-graining scale and join the IR parts.
Calling them $\xi_Q^I$ and $\xi_{\tilde{P}I}$ as
\bae{
	\xi_Q^I(x)=-\int\dk\ee^{i\kdx}\dot{W}(k/k_\sigma)\phi^I(t,\mathbf{k}), \quad\quad 
	\xi_{\tilde{P}I}(x)=-\int\dk\ee^{i\kdx}\dot{W}(k/k_\sigma)\tilde{\pi}_I(t,\mathbf{k}),
}
the covariant EoM for IR modes is given by
\bae{\label{eq: Langevin in heuristic}
	\dot{\varphi}^I=\frac{N_\IR}{a^3}G^{IJ}\varpi_J+\xi_Q^{I}, \quad\quad
	D_t\varpi_I=-a^3N_\IR V_I+\xi_{\tilde{P}I}.
}
These $\xi$ are nothing but the classical noise for coarse-grained fields and make them Brownian motions.

The interpretation of $\xi$ terms can be suggested by their statistics. If one adopts the step window function $W(k/k\sigma)=\theta(k-k_\sigma)$,
$\xi$'s quantum correlators can be calculated as
\bae{\label{eq: noise correlation}
	\bce{
		\dps
		\braket{\xi_Q^I}=\braket{\xi_{\tilde{P}I}}=0, \\[10pt]
		\dps
		\braket{\xi_Q^I(x)\xi_Q^J(x^\prime)}
		=\calP_{QQ}{}^{IJ}(k_\sigma)\frac{\dot{k}_\sigma}{k_\sigma}\delta(t-t^\prime)\frac{\sin k_\sigma r}{k_\sigma r},
		\\[10pt]
		\dps
		\braket{\xi_Q^I(x)\xi_{\tilde{P}J}(x^\prime)}=\braket{\xi_{\tilde{P}J}(x)\xi_Q^I(x^\prime)}^*
		=\calP_{Q\tilde{P}}{}^I{}_J(k_\sigma)\frac{\dot{k}_\sigma}{k_\sigma}\delta(t-t^\prime)
		\frac{\sin k_\sigma r}{k_\sigma r}, \\[10pt]
		\dps
		\braket{\xi_{\tilde{P}I}(x)\xi_{\tilde{P}J}(x^\prime)}=\calP_{\tilde{P}\tilde{P}IJ}(k_\sigma)\frac{\dot{k}_\sigma}{k_\sigma}\delta(t-t^\prime)
		\frac{\sin k_\sigma r}{k_\sigma r},
	}
}
where $r=|\mathbf{x}-\mathbf{x}^\prime|$ and
\bae{
	(2\pi)^3\delta^{(3)}(\mathbf{k}+\mathbf{k}^\prime)\frac{k^3}{2\pi^2}\calP_{QQ}{}^{IJ}(k)=\braket{\phi^I(t,\mathbf{k})\phi^J(t,\mathbf{k}^\prime)},
}
and so on. Since the IR fields are treated as classical backgrounds in the stochastic formalism, $\xi$ would be also interpreted as 
classical but random variables whose statistics inherits the above quantum correlators. Their correlations are suppressed by
$\sin k_\sigma r/k_\sigma r$ beyond the coarse-graining scale. Also the noise follows the Gaussian distribution since we here consider
only the linear perturbation theory.
Therefore $\xi$ is regarded as white ($\propto\delta(t-t^\prime)$) and $k_\sigma$-patch-independent
Brownian noise. This is the heuristic derivation of the Langevin EoM in the stochastic formalism.

Let us mention the reality of the noise terms.
Since $\varphi^I$ and $\varpi_I$ are real scalar fields, the noise terms should be also real so that Eq.~(\ref{eq: Langevin in heuristic})
can be interpreted as a proper Langevin equation. However the noise correlations~(\ref{eq: noise correlation}) do not ensure their reality at all.
Therefore, to apply the stochastic formalism, all imaginary parts of noise correlations should be at least suppressed by 
the gradient expansion $\sigma$. In the massless limit, the imaginary parts are indeed suppressed by $\calO(\sigma^3)$ as we will see
in Sec.~\ref{sec: UV EoM}. In more general cases, one has to carefully check whether imaginary parts can be neglected or not.%\footnote{The 
%imaginary parts correspond with the antisymmetric component of the noise correlations as can be seen in Eq.~(\ref{eq: noise correlation}).
%We will see in Sec.~\ref{} that the Fokker-Planck equation equivalent to our Langevin equation~(\ref{eq: Langevin in heuristic}) depends only on
%the symmetric part of the noise correlations. Therefore the assumption that the IR modes are well-classicalized, or equivalently,
%the observables can be described well by the Fokker-Planck equation can be validated only if the antisymmetric part is negligible.
%It again suggests that the negligiblity of the imaginary part of noise is one necessary condition for the stochastic formalism.}

We also mention the energy constraint~(\ref{eq: energy const}). It reads in the stochastic formalism
\bae{
	-\frac{3\Mpl^2\calH^2}{N_\IR^2}+\frac{G^{IJ}}{2a^6}\varpi_I\varpi_J+V=-\frac{6\Mpl^2\calH^2}{N_\IR^3}\alpha
	-\frac{2\Mpl^2\calH}{N_\IR^2}\frac{\nabla^2}{a^2}\psi-\frac{G^{IJ}}{a^6}\varpi_IP_J-\frac{G^{IJ}{}_{,K}}{2a^6}\varpi_I\varpi_JQ^K-V_IQ^I.
}
Because it does not include any time derivative, it can be decomposed completely into IR and UV parts by taking its spatial derivative and applying the 
long-wavelength limit $\sigma\to0$ as 
\bae{\label{eq: perturbation energy const}
	\bce{
		\dps
		-\frac{3\Mpl^2\calH^2}{N_\IR^2}+\frac{G^{IJ}}{2a^6}\varpi_I\varpi_J+V=0, \\[10pt]
		\dps
		-\frac{6\Mpl^2\calH^2}{N_\IR^3}\alpha-\frac{2\Mpl^2\calH}{N_\IR^2}\frac{\nabla^2}{a^2}\psi
		-\frac{G^{IJ}}{a^6}\varpi_IP_J-\frac{G^{IJ}{}_{,K}}{2a^6}\varpi_I\varpi_JQ^K-V_IQ^I=0.
	}
}
The first one is nothing but the standard Friedmann constraint. Therefore it is suggested that the Friedmann constraint does hold in each 
$k_\sigma$-patch
even in the stochastic formalism. It will be checked also in the effective action approach. Note that, if one converts $\varpi_I$ to $\dot{\varphi}^I$
with use of the IR EoM~(\ref{eq: Langevin in heuristic}), the Friedmann constraint necessarily includes the noise term.
That is one of the reasons why the Hamiltonian language is preferred to the Lagrangian one in the stochastic formalism.







%\begin{itemize}
%\item introduce our notation
%\item derive the Langevin eq. in the EoM approach
%\item linear theory for UV but non-perturbative for IR
%\end{itemize}


%%%%%%%%%%%%%%%%%%%%%%%%%%%%%%%
\begin{comment}
The basic idea of the stochastic formalism is that the inflaton fields coarse-grained on a superhorizon scale should be interpreted as
the classical background fields rather than the homogeneous modes because the perturbations exiting the horizon will be classicalized by
some mechanism and unable to be distinguished from the homogeneous modes.
Following this hypothesis, one would derive the classical equation of motion (EoM) for the coarse-grained fields in the stochastic formalism.
In this section, we simply decompose the original scalar EoM into the coarse-grained part and the short-wavelength perturbation
to imply the conclusive equation.

Since we take into account the metric sector and then several constraint equations appear,
the Hamiltonian language would be useful. The same approach can be seen in Refs.~\cite{Habib:1992ci,Grain:2017dqa}.
Adopting the ADM formalism for the metric
\bae{
	\dd s^2=-N^2\dd t^2+\gamma_{ij}\left(\dd x^i+\beta^i\dd t\right)\left(\dd x^j+\beta^j\dd t\right).
}
one can expand the action as~\cite{Salopek:1990jq}
\bae{
	S&=\int\dd^4x\sqrt{-g}\left[\frac{1}{2}\Mpl^2R-\frac{1}{2}G_{IJ}(\phi)g^{\mu\nu}\partial_\mu\phi^I\partial_\nu\phi^J-V(\phi)\right] \nonumber \\
	&=\int\dd^4xN\sqrt{\gamma}\left[\frac{\Mpl^2}{2}\left(K_{ij}K^{ij}-K^2\right)-G_{IJ}g^{\mu\nu}\partial_\mu\phi^I\partial_\nu\phi^J-V\right],
}
where
\bae{
	K_{ij}=\frac{1}{2N}(\beta_{i,j}+\beta_{j,i}-2a^2\calH\delta_{ij}), \quad\quad K=K^i_i.
}
The spatial indices $i$ and $j$ are raised and lowered by the spatial metric $\gamma_{ij}=a^2\delta_{ij}$ and $\gamma^{ij}=a^{-2}\delta^{ij}$.
Commas and dots represent spatial and time derivatives and the calligraphy $\calH=\dot{a}/a$ stands for the generalized Hubble parameter
for arbitrary $N$, while the normal $H$ is left for the standard Hubble parameter with $N=1$.
Capital Latin indices $I$ and $J$ label the scalar fields and the field space metric $G_{IJ}(\phi)$ is not flat in general.
\end{comment}
%%%%%%%%%%%%%%%%%%%%%%%%%%%%%%%






\section{Effective Hamiltonian action}\label{sec: effective action}

The purpose of the stochastic formalism is describing the dynamics of coarse-grained fields. 
Therefore it may come to one's mind to construct an effective theory for coarse-grained fields like as renormalization.
That is, by integrating out the UV modes, one can obtain the effective action for IR modes and the correction terms from UV parts 
are expected to represent the noise contribution. In cosmology, the expectation values of various operators are often calculated for 
an in-state known as the Bunch-Davies vacuum for example, instead of the in-out scattering amplitude as often calculated in particle physics.
In this case, the partition function for in-in expectation values can be formulated by the path integral along with a closed time path (CTP) from the in-state
to the in-state again via a sufficient future time~\cite{Schwinger:1960qe,Feynman:1963fq}.
Several authors~\cite{Morikawa:1989xz,Matarrese:2003ye,Levasseur:2013ffa,Moss:2016uix} formulated the stochastic inflation 
with use of this CTP formalism in terms of the Lagrangian where the independent variables are $\phi$ and $\dot{\phi}$.
However, in the stochastic formalism, $\phi$ is a stochastic process and its time-derivative $\dot{\phi}$ is not mathematically well-defined.
Also, metric perturbations being included as generally required, the constraint equations will be much clear in the Hamiltonian language.
Therefore in this section we derive the Langevin equations by constructing the effective action in the Hamiltonian formalism.

%The noise identification in the previous section seems to be relatively ad hoc and unclear.
%In the stream of the derivation of the quantum Brownian motion in the path integral approach,
%Morikawa~\cite{Morikawa:1989xz} proposed the effective action approach to obtain the Langevin equation for IR modes 
%by integrating out UV modes in the in-in or closed time path (CTP) formalism~\cite{Schwinger:1960qe,Feynman:1963fq}.
%After that, several authors followed this approach~\cite{Matarrese:2003ye,Levasseur:2013ffa,Moss:2016uix}.
%Though their analyses were in terms of the Lagrangian, recently Tokuda and Tanaka~\cite{Tokuda:2017fdh} showed the effective action with use of
%the Hamiltonian language in the single field without metric perturbations, which we extend in this paper.

We again start by the general multi-scalar action~(\ref{eq: original action}) 
\bae{
	S=\int\dd^4x\sqrt{-g}\left[\frac{1}{2}\Mpl^2\calR-\frac{1}{2}g^{\mu\nu}G_{IJ}(\phi)\partial_\mu\phi^I\partial_\nu\phi^J-V(\phi)\right],
}
in the flat-gauge ADM formalism~(\ref{eq: ADM})
\bae{
	\dd s^2=-N^2\dd t^2+a^2\mathblue{\delta_{ij}}(\dd x^i+\beta^i\dd t)(\dd x^j+\beta^j\dd t).
}
In terms of the Hamiltonian, the action can be written as
\bae{
	S=\int\dd^4x(\pi^{ij}\dot{\gamma}_{ij}+\pi_I\dot{\phi}^I-\scrH),
}
where the Hamiltonian density is given by Eq.~(\ref{eq: Hamiltonian}).
Note that here $\dot{\gamma}_{ij}$ and $\dot{\phi}^I$ are not independent variables but functions of $\phi^I$, $\pi_I$, and so on,
given by the solutions of the Legendre transformation~(\ref{eq: Legendre trs}).

{\bf SRP: Actually, in the path integral formulation that we will ue below, we will not (and we can not) use these on-shell relations}

Because we only consider the linear perturbation for UV modes throughout this paper, we expand the action
up to quadratic order in them as
\bae{
	S\simeq&\,S^{(0)}+S^{(1)}+S^{(2)}, \\
	S^{(0)}=&\int\dd^4x\left[\varpi_I\dot{\varphi}^I-a^3N_\IR\left(\frac{1}{2a^6}G^{IJ}\varpi_I\varpi_J+V+\frac{3\Mpl^2\calH^2}{N_\IR^2}\right)\right], \label{eq: S0} \\
	S^{(1)}=&\int\dd^4x\left[P_I\left(\dot{\varphi}^I-\frac{N_\IR}{a^3}G^{IJ}\varpi_J\right)
	-Q^I\left(\dot{\varpi}_I+\frac{N_\IR}{2a^3}G^{JK}{}_{,I}\varpi_J\varpi_K+a^3N_\IR V_I\right) \right. \nonumber \\
	&\left.-a^3\alpha\left(\frac{1}{2a^6}G^{IJ}\varpi_I\varpi_J+V-\frac{3\Mpl^2\calH^2}{N_\IR^2}\right)\right], \label{eq: S1} \\
	S^{(2)}=&\int\dd^4x\left[-\frac{3\Mpl^2\calH^2a^3}{N_\IR^3}\alpha^2
	-a^3\alpha\left(\frac{1}{a^6}G^{IJ}\varpi_IP_J+\frac{1}{2a^6}G^{IJ}{}_{,K}Q^K\varpi_I\varpi_J+V_IQ^I
	+\frac{2\Mpl^2\calH}{N_\IR^2}\frac{\nabla^2}{a^2}\psi\right) \right. \nonumber \\
	&\left. +\varpi_IQ^I\frac{\nabla^2}{a^2}\psi-a^3N_\IR\left(\frac{1}{2a^6}G^{IJ}P_IP_J+\frac{1}{a^6}G^{IJ}{}_{,K}Q^K\varpi_IP_J
	+\frac{1}{4a^6}G^{IJ}{}_{,KL}\varpi_I\varpi_JQ^KQ^L \right.\right. \nonumber \\
	&\left.\left. -\frac{1}{2}G_{IJ}Q^I\frac{\nabla^2}{a^2}Q^J+\frac{1}{2}V_{IJ}Q^IQ^J\right)+P_I\dot{Q}^I\right]. \label{eq: S2}
}
Note that if one neglects the UV modes and consider only $S^{(0)}$, its variation (formally equivalent to the coefficient of $Q^I$, $P_I$, and $\alpha$
in $S^{(1)}$) reproduces the classical EoM~(\ref{eq: classical EoM}) as
\bae{\label{eq: classical IR EoM}
	\dot{\varphi}^I\approx\frac{N_\IR}{a^3}G^{IJ}\varpi_J, \quad\quad
	D_t\varpi_I\approx-a^3N_\IR V_I, \quad\quad
	\frac{3\Mpl^2\calH^2}{N_\IR^2}\approx\frac{1}{2a^6}G^{IJ}\varpi_I\varpi_J+V,
}
which hold only if the noise terms are neglected. Actually it will be shown that the last Friedmann equation does hold even under the existence 
of noise.

Here let us mention that this action can be covariantized as
\bae{
	S^{(0)}=&\int\dd^4x\left[\varpi_I\dot{\varphi}^I-a^3N_\IR\left(\frac{1}{2a^6}\varpi_I\varpi^I+V+\frac{3\Mpl^2\calH^2}{N_\IR^2}\right)\right], \\
	S^{(1)}=&\int\dd^4x\left[\tilde{P}_I\left(\dot{\varphi}^I-\frac{N_\IR}{a^3}\varpi^I\right)-Q^I\left(D_t\varpi_I+a^3N_\IR V_I\right) \right. \nonumber \\
	&\left.-a^3\alpha\left(\frac{1}{2a^6}\varpi_I\varpi^I+V-\frac{3\Mpl^2\calH^2}{N_\IR^2}\right)\right], \\
	S^{(2)}=&\int\dd^4x\left[-\frac{3\Mpl^2\calH^2a^3}{N_\IR^3}\alpha^2-a^3\alpha\left(\frac{1}{a^6}\varpi_I\tilde{P}^I+V_IQ^I
	+\frac{2\Mpl^2\calH}{N_\IR^2}\frac{\nabla^2}{a^2}\psi\right)+\varpi_IQ^I\frac{\nabla^2}{a^2}\psi \right. \nonumber \\
	&\left.-a^3N_\IR\left(\frac{1}{2a^6}\tilde{P}_I\tilde{P}^I-\frac{1}{2}Q_I\frac{\nabla^2}{a^2}Q^I+\frac{1}{2}V_{I;J}Q^IQ^J-
	\frac{1}{2a^6}R_I{}^{JK}{}_L\varpi_J\varpi_KQ^IQ^L\right)
	+\tilde{P}_ID_tQ^I\right], \label{eq: S2 cov}
}
where
\bae{
	\tilde{P}_I=P_I-\Gamma^K_{IJ}\varpi_KQ^J,
}
and $R_{IJKL}$ is a Riemann tensor corresponding with the field space metric $G_{IJ}$.
Also $D_t$ and a semicolon represent the covariant derivatives defined by
\bae{
	D_t\varpi_I=\dot{\varpi}_I-\Gamma^K_{IJ}\dot{\varphi}^J\varpi_K, \quad\quad D_tQ^I=\dot{Q}^I+\Gamma^I_{JK}\dot{\varphi}^JQ^K,
	\quad\quad V_{I;J}=\partial_JV_I-\Gamma^K_{IJ}V_K.
}
Field space indices $I$, $J$, and so on are raised and lowered by $G_{IJ}$.
The introduction of $\tilde{P}_I$ means that the original $P_I$ is not transformed as a vector since it is related with a non-vector $\dot{Q}_I$, 
but instead $\tilde{P}_I=\delta S^{(2)}/\delta (D_tQ^I)$ is a correct vector. 
Note that one has to use the classical EoM~(\ref{eq: classical IR EoM}) to derive the expression
of $S^{(2)}$. This EoM holds only up to noise as we mentioned, 
but the noise term itself comes from the UV modes and therefore the difference will be the cubic order
in the UV modes which can be neglected here.
Rather than this covariant action, the following calculations are based on the original action~(\ref{eq: S0})--(\ref{eq: S2}) because they contain less time derivatives and therefore the calculations will be clearer without subtleties.
However, after covariantizing the final expressions, the correct vector $\tilde{P}_I$ will appear instead of $P_I$.

As we mentioned in the previous section, 
the strong point of the ADM formalism is that the metric components $N$ and $\beta$ appear as mere Lagrangian multipliers in the action.
Therefore they can be integrated out in advance, which is equivalent
to substituting the solutions of the corresponding Euler-Lagrange constraints.
In fact one needs only the momentum constraint
\bae{\label{eq: alpha}
	\var{S}{\psi}=0, \quad \Leftrightarrow \quad \alpha=\frac{N_\IR^2} {2\Mpl^2\calH a^3}\varpi_IQ^I,
}
and then both $\alpha$ and $\psi$ can be eliminated. Also we define the IR and UV modes orthogonally so that 
the integral of IR and UV parts of some variables vanishes:\footnote{If an integral includes two or more than two IR operators like as 
$\int\dd^4x\,\varphi\varphi Q$, the convolution between the multiple IR operators may lead to the short wavelength mode and the integral does not vanish.
In this paper, we neglect these mode coupling terms for simplicity to see only the effect of the horizon crossing of UV modes.
We keep the terms with two UV modes. \label{footnote: orthogonal}}
\bae{
	\int\dd^4x\,\calO_\IR(x)\scrO_\UV(x)=0.
}
Then almost all terms in $S^{(1)}$ can be dropped except for time-derivative terms. The IR and UV modes can be assumed to be coupled 
through the time derivative of the window function and indeed it is nothing but the source of the noise for IR modes.
Finally the considered action can be summarized as%\footnote{Note that we omit the term $\Gamma^K_{IJ}\dot{\varphi}^IQ^J\varpi_K$
%even though it includes a time derivative because there is a non-derivative IR part $\Gamma^K_{IJ}\varpi_K$ and then the group of
%$\Gamma^K_{IJ}\dot{\varphi}^I\varpi_K$ can be also interpreted as an IR part similarly to the assumption in footnote~\ref{footnote: orthogonal}.}
\bae{\label{eq: quadratic action}
	S\simeq S^{(0)}+\int\dd^4x\left[-\varphi_{\bar{X}I}\dot{Q}_{X}^I+\frac{1}{2}Q_{X}^I\Lambda_{XYIJ}Q_Y^J\right].
}
Here we introduced $X$ and $Y$ indices taking $Q$ or $P$ as
\bae{
	Q_Q^I=Q^I, \quad\quad Q_{P}^I=P_I.
}
Also bar indices $\bar{X}$ and $\bar{Y}$ are defined by
\bae{
	\varphi_{\bar{Q}I}=-\varpi_I, \quad\quad \varphi_{\bar{P}I}=\varphi^I.
}
The concrete expression of $\Lambda$ is given in Sec.~\ref{sec: UV EoM}.
Note that the position of $I$ and $J$ is taken so as to correspond with $Q$
in the $XY$ notation, but it is implicitly taken correctly also for $P$-components
(e.g. $\Lambda_{XYIJ}=(\Lambda_{QQIJ},\Lambda_{QPI}{}^J,\Lambda_{QP}{}^I{}_J,\Lambda_{PP}{}^{IJ})$).




\subsection{Effective action}

Now let us formulate the path integral for the in-in vacuum expectation values described by the action~(\ref{eq: quadratic action}).
Though the calculation of the scattering process, that is, the expectation value between the in- and out-state is quite common,
what we want now is that between the in- and in-state. In such a situation, one generally considers the problem after reducing it to
that of the scattering process.
In other words, one can interpret the in-in one as the scattering process on the closed time path from the in- to out- and then in-state again.
The partition function is then calculated on the closed time path $C=C^++C^-$ as
\bae{
	Z[\calJ_{XI},\scrJ_{XI}]&=\calN\int_C\scrD\varphi_X^I\scrD Q_X^I\ee^{i\left[S+\int\dd^4x\left(\calJ_{XI}\varphi_X^I+\scrJ_{XI}Q_X^I\right)\right]},
}
where $C^+$ and $C^-$ are the time paths from the in- to out-state and from the out- to in-state respectively.
$\mathcal{N}$ formally denotes the field independent normalization factor which does not necessarily take the same value
in differential equations.
If one expresses the field on $C^{+/-}$ as $\varphi_X^{I+/-}$ and $Q_X^{I+/-}$, the partition function can be simplified to the standard 
scattering process with doubled d.o.f. as
\bae{
	&Z[\calJ_{XI}^\pm,\scrJ_{XI}^\pm]=\calN\int_{t=-\infty\to\infty}\scrD\varphi_X^{I\pm}\scrD Q_X^{I\pm} \nonumber \\
	&\times\exp\left[S[\varphi_X^{I+},Q_X^{I+}]
	-S[\varphi_X^{I-},Q_X^{I-}]+\int\dd^4x\left(\calJ_{XI}^+\varphi_X^{I+}+\scrJ_{XI}^+Q_X^{I+}-\calJ_{XI}^-\varphi_X^{I-}
	-\scrJ_{XI}^-Q_X^{I-}\right)\right].
}
Introducing the matrix derivative operator
\bae{
	\tilde{\Lambda}_{XYIJab}(x,y)=\bpme{
		\Lambda_{XYIJ}(\varphi^+(x),\varpi^+(x)) & 0 \\
		0 & -\Lambda_{XYIJ}(\varphi^-(x),\varpi^-(x))
	}\delta^{(4)}(x-y),
}
one can write
\bae{\label{eq: quadratic Z}
	Z[\calJ_{XIa},\scrJ_{XIa}]\simeq&\,\calN\int\scrD\varphi_X^{Ia}\ee^{i(S^{(0)+}-S^{(0)-}+\int\dd^4x\calJ_{XIa}\varphi_X^{Ia})} \nonumber \\
	&\times\int\scrD Q_X^{Ia}\exp\left[i\int\dd^4x\left(-\varphi_{\bar{X}Ia}\dot{Q}_{X}^{Ia}+\frac{1}{2}Q_{X}^{Ia}\tilde{\Lambda}_{XYIJab}Q_X^{Jb}
	+\scrJ_{XIa}Q_X^{Ia}\right)\right].
}
Here indices $a$ and $b$ running over $+$ and $-$ are raised and lowered by a two-dimensional metric $\sigma_{3ab}=\sigma_3^{ab}=\diag(1,-1)$.

Now let us consider the role of $\varphi_{\bar{X}Ia}\dot{Q}_{X}^{Ia}$ term. In the definition with the window function as the previous section,
$\dot{Q}$ includes a time derivative of the window function and it is expected to represent the effect of the horizon crossing of UV modes.
Since $\dot{W}(k/\sigma aH)=-\delta(t-t_\sigma(k))$ where the mode $k$ crosses the cutoff scale $\sigma aH$ at $t_\sigma(k)$,
this term can be interpreted as the delta-function coupling:
\bae{\label{eq: IR-UV coupling}
	-\int\dd^4x\varphi_{\bar{X}Ia}(t,\mathbf{x})\dot{Q}_{X}^{Ia}(t,\mathbf{x})\to
	\int\dd t\int\dk\delta(t-t_\sigma(k))\varphi_{\bar{X}Ia}(t,\mathbf{k})Q_{X}^{Ia}(t,-\mathbf{k}).
}
If one moves to the useful Keldysh basis (we will mention its meaning later) defined by
\bae{\label{eq: Keldysh}
	\bpme{
		\bar{\varphi}_X^I \\
		\varphi_X^{I\Delta}
	}=\bpme{
		1/2 & 1/2 \\
		1 & -1
	}\bpme{
		\varphi_X^{I+} \\
		\varphi_X^{I-}
	}, \quad \Leftrightarrow \quad
	\bpme{
		\varphi_X^{I+} \\
		\varphi_X^{I-}
	}=\bpme{
		1 & 1/2 \\
		1 & -1/2
	}\bpme{
		\bar{\varphi}_X^I \\
		\varphi_X^{I\Delta}
	},
}
it can be expressed by two parts as
\bae{\label{eq: IR-UV in Keldysh}
	\int\dd t\int\dk\delta(t-t_\sigma(k))\varphi_{\bar{X}Ia}Q_{X}^{Ia}=
	\int\dd t\int\dk\delta(t-t_\sigma(k))\left(\varphi_{\bar{X}I}^{\Delta}\bar{Q}_{X}^I+\bar{\varphi}_{\bar{X}I}Q_{X}^{I\Delta}\right).
}
However the latter part causes hateful dissipation and mass renormalization terms as we will see. 
It alters IR propagators even in the free test field case.
These terms had been simply neglected in the literature~\cite{Morikawa:1989xz,Matarrese:2003ye,Levasseur:2013ffa}, assumed to be small.
But in fact their effects are as large as those of the noise terms and therefore they cannot be neglected at the perturbation order.
Recently Moss and Rigopoulos~\cite{Moss:2016uix} casted a doubt 
that simple decomposition of fields and integration measures could recover the original theory
when the window is time-dependent. Also Tokuda and Tanaka claimed in their paper~\cite{Tokuda:2017fdh} that
the IR-UV coupling should be determined to reproduce the original propagator rather than derived by the straightforward decomposition.
Following them, we will omit the latter term by hand. For a while we just introduce a fictitious coefficient $\epsilon$ and keep it to see
its effect, that is, we adopt the following IR-UV coupling:
\bae{
	\int\dd t\int\dk\delta(t-t_\sigma(k))\left(\varphi_{\bar{X}I}^{\Delta}\bar{Q}_{X}^I+\epsilon \bar{\varphi}_{\bar{X}I}Q_{X}^{I\Delta}\right)
	&=\int\dd t\int\dk\delta(t-t_\sigma(k))\varphi_{\bar{X}Ia}A^a{}_bQ_{X}^{Jb} \nonumber \\
	&=:\int\dd^4x\tilde{\varphi}_{\bar{X}Ia}Q_{X}^{Ia}
}
where
\bae{
	A^a{}_b=\frac{1}{2}\left[\bpme{
		1 & -1 \\
		-1 & 1	
	}+\epsilon\bpme{
		1 & 1 \\
		1 & 1
	}\right], \quad\quad \tilde{\varphi}_{\bar{X}Ia}=\int\dk\ee^{-i\mathbf{k}\cdot{\mathbf{x}}}\delta(t-t_\sigma)\varphi_{\bar{X}Ib}A^b{}_a.
}

The partition function~(\ref{eq: quadratic Z}) is quadratic in $Q$. Therefore it can be easily integrated over $Q$, giving rise to
\bae{
	Z[\calJ_{XIa},\scrJ_{XIa}]=&\,\calN\int\scrD\varphi_X^{Ia}\ee^{i(S^{(0)+}-S^{(0)-}+\int\dd^4x\calJ_{XIa}\varphi_X^{Ia})} \nonumber \\
	&\times\exp\left[-\frac{i}{2}\int\dd^4x\dd^4x^\prime
	\left(\tilde{\varphi}_{\bar{X}Ia}+\scrJ_{XIa}\right)(\tilde{\Lambda}^{-1})_{XY}{}^{IJab}\left(\tilde{\varphi}_{\bar{Y}Jb}+\scrJ_{YJb}\right)\right].
}
Finally we make a key assumption that the IR modes are well classicalized so that the integration with respect to $\varphi$ simply leads to
the replacement of $\varphi$ to its vacuum expectation value (vev) $\braket{\varphi}_\calJ$ under the existence of the source term $\calJ$.
$\braket{\varphi}_\calJ$ includes all quantum effects and therefore its dynamics will be written by the Langevin equation that we want.
The partition function after the field integration reads
\bae{\label{eq: Z after integration}
	Z[\calJ_{XIa},\scrJ_{XIa}]\simeq&\,\calN\exp\left[i\left(S^{(0)}[\braket{\varphi_X^{I+}}_\calJ]-S^{(0)}[\braket{\varphi_X^{I-}}_\calJ]
	+\int\dd^4x\calJ_{XIa}\braket{\varphi_X^{Ia}}_\calJ \right.\right. \nonumber \\
	&\left.\left. -\frac{1}{2}\int\dd^4x\dd^4x^\prime
	\left(\braket{\tilde{\varphi}_{\bar{X}Ia}}_\calJ+\scrJ_{XIa}\right)(\tilde{\Lambda}^{-1})_{XY}{}^{IJab}
	\left(\braket{\tilde{\varphi}_{\bar{Y}Jb}}_\calJ+\scrJ_{YJb}\right) 
	\right)\right] \nonumber \\
	=:&\,\calN\ee^{iW[\calJ_{XIa},\scrJ_{XIa}]}.
}
Then the effective action $\Gamma$ is given by the Legendre transformation of $W$ as
\bae{
	\Gamma[\braket{\varphi_X^{Ia}},\braket{Q_X^{Ia}}]=W[\calJ_{XIa},\scrJ_{XIa}]-\calJ_{XIa}\braket{\varphi_X^{Ia}}-\scrJ_{XIa}\braket{Q_X^{Ia}}.
}
Here the field vev and the source term are related by
\bae{
	\braket{\varphi_X^{Ia}}=\var{W}{\calJ_{XIa}}, \quad\quad \braket{Q_X^{Ia}}=\var{W}{\scrJ_{XIa}}=-\int\dd^4x^\prime
	(\tilde{\Lambda}^{-1})_{XY}{}^{IJab}
	(\braket{\tilde{\varphi}_{\bar{Y}Jb}}_\calJ+\scrJ_{YJb}).
}
Therefore, by substituting inverse solutions for $\scrJ$, the explicit form of the effective action reads
\bae{
	\Gamma[\braket{\varphi_X^{Ia}},\braket{Q_X^{Ia}}]=S^{(0)+}-S^{(0)-}+\frac{1}{2}\int\dd^4x\dd^4x^\prime\braket{Q_X^{Ia}}
	\tilde{\Lambda}_{XYIJab}\braket{Q_Y^{Jb}}+\int\dd^4x\braket{\tilde{\varphi}_{\bar{X}Ia}}\braket{Q_X^{Ia}}.
}
The vev of $Q$ can be solved as a stationary point of the effective action:
\bae{
	\var{\Gamma}{\braket{Q_X^{Ia}}}=0, \quad \Leftrightarrow \quad
	\braket{Q_X^{Ia}}=-\int\dd^4x^\prime(\tilde{\Lambda}^{-1})_{XY}{}^{IJab}\braket{\tilde{\varphi}_{\bar{X}Jb}}.
}
Substituting it back into the effective action, one finally obtains the effective action for IR modes as
\bae{\label{eq: GammaIR}
	\Gamma_\IR[\braket{\varphi_X^{Ia}}]=S^{(0)+}-S^{(0)-}-\frac{1}{2}\int\dd^4x\dd^4x^\prime
	\braket{\tilde{\varphi}_{\bar{X}Ia}}(\tilde{\Lambda}^{-1})_{XY}{}^{IJab}
	\braket{\tilde{\varphi}_{\bar{Y}Jb}}.
}
Particularly the last term $\SIA=-\frac{1}{2}\int\dd^4x\dd^4x^\prime
\braket{\tilde{\varphi}_{\bar{X}Ia}}(\tilde{\Lambda}^{-1})_{XY}{}^{IJab}\braket{\tilde{\varphi}_{\bar{Y}Jb}}$
is called \emph{influence action}, which includes the effect of the horizon-exiting UV modes.
We will omit brackets hereafter for conciseness.





\subsection{Influence action}\label{sec: influence action}

In the influence action, $\tilde{\Lambda}^{-1}$ has not been clarified yet. By taking the variation of the partition function~(\ref{eq: Z after integration}),
one finds that $\tilde{\Lambda}^{-1}$ is indeed nothing but the time-ordered two point function of $Q$ as
\bae{
	\braket{T_C\hat{Q}_X^{Ia}(x)\hat{Q}_Y^{Jb}(x^\prime)}
	=\left.\frac{1}{i}\var{}{\scrJ_{XIa}(x)}\frac{1}{i}\var{}{\scrJ_{YJb}(x^\prime)}Z\right|_{\calJ=\scrJ=0}
	=i(\tilde{\Lambda}^{-1})_{XY}{}^{IJab}(x,x^\prime).
}
Here the time ordering should be defined along with the closed time path $C$, that is,
\bae{
	T_C\hat{Q}_{X}^{Ia}(x)\hat{Q}_Y^{Jb}(x^\prime)=
	\bce{
		\dps
		\theta(t-t^\prime)\hat{Q}_{X}^{Ia}(x)\hat{Q}_Y^{Jb}(x^\prime)+\theta(t^\prime-t)\hat{Q}_Y^{Jb}(x^\prime)\hat{Q}_X^{Ia}(x), & (a=+,\,b=+), \\
		\dps
		\hat{Q}_Y^{Jb}(x^\prime)\hat{Q}_X^{Ia}(x), & (a=+,\,b=-), \\
		\dps
		\hat{Q}_X^{Ia}(x)\hat{Q}_Y^{Jb}(x^\prime), & (a=-,\,b=+), \\
		\dps
		\theta(t^\prime-t)\hat{Q}_{X}^{Ia}(x)\hat{Q}_Y^{Jb}(x^\prime)+\theta(t-t^\prime)\hat{Q}_Y^{Jb}(x^\prime)\hat{Q}_X^{Ia}(x), & (a=-,\,b=-).
	}
}
The field operator $\hat{Q}_X^{Ia}$ can be expanded as
\bae{\label{eq: mode expansion}
	\hat{Q}_X^{Ia}(x)=\int\dk\left[Q_{X\alpha}^{Ia}(t,k)\hat{a}_\alpha(\mathbf{k})\ee^{i\mathbf{k}\cdot\mathbf{x}}
	+Q_{X\alpha}^{*Ia}(t,k)\hat{a}^\dagger_\alpha(\mathbf{k})\ee^{-i\mathbf{k}\cdot\mathbf{x}}\right],
}
with the annihilation-creation operators
\bae{\label{eq: aadagger}
	[\hat{a}_\alpha(\mathbf{k}),\hat{a}^\dagger_\beta(\mathbf{q})]=(2\pi)^3\delta_{\alpha\beta}\delta^{(3)}(\mathbf{k}-\mathbf{q}).
}
Here indices $\alpha$ and $\beta$ take as many values as the field index $I$. The mode function $Q_{X\alpha}^{Ia}(t,k)$ can be given by
the solution $Q_{X\alpha}^I(t,k)$ of the UV EoM as $Q_{X\alpha}^{I+}=Q_{X\alpha}^{I-}=Q_{X\alpha}^I$.\footnote{Since
the coefficients $\calH$, $V_{IJ}$, and so on in the EoM depend on the background IR mode $\varphi_X^{I\pm}$,
the solutions $Q_{X\alpha}^{I+}$ and $Q_{X\alpha}^{I-}$ are generally different. However we will consider the IR EoM at 
$\varphi_X^{I+}=\varphi_X^{I-}$ and therefore these two solutions can be given by the same $Q_{X\alpha}^I$.}
Then with use of the mode function, the propagator $\tilde{\Lambda}^{-1}$ can be written as
\bae{
	&(\tilde{\Lambda}^{-1})_{XY}{}^{IJab}(x,x^\prime)=-i\braket{T_C\hat{Q}_X^{Ia}(x)\hat{Q}_Y^{Jb}(x^\prime)} \nonumber \\
	&=\int\dk\bpme{
		\begin{array}{c}
			\theta(t-t^\prime)Q_{X\alpha}^I(t,k)Q^{*J}_{Y\alpha}(t^\prime,k)\ee^{i\mathbf{k}\cdot(\mathbf{x}-\mathbf{x}^\prime)} \\
			+\theta(t^\prime-t)Q_{Y\alpha}^J(t^\prime,k)Q_{X\alpha}^{*I}(t,k)\ee^{i\mathbf{k}\cdot(\mathbf{x}^\prime-\mathbf{x})}
		\end{array}
		& Q^J_{Y\alpha}(t^\prime,k)Q_{X\alpha}^{*I}(t,k)\ee^{i\mathbf{k}\cdot(\mathbf{x}^\prime-\mathbf{x})} \\
		Q_{X\alpha}^I(t,k)Q^{*J}_{Y\alpha}(t^\prime,k)\ee^{i\mathbf{k}\cdot(\mathbf{x}-\mathbf{x}^\prime)} &
		\begin{array}{c}
			\theta(t^\prime-t)Q_{X\alpha}^I(t,k)Q^{*J}_{Y\alpha}(t^\prime,k)\ee^{i\mathbf{k}\cdot(\mathbf{x}-\mathbf{x}^\prime)} \\
			+\theta(t-t^\prime)Q^J_{Y\alpha}(t^\prime,k)Q_{X\alpha}^{*I}(t,k)\ee^{i\mathbf{k}\cdot(\mathbf{x}^\prime-\mathbf{x})}
		\end{array}
	}
}

Then, moving to the Keldysh basis~(\ref{eq: Keldysh}), one can summarize the influence action as
\bae{\label{eq: SIA}
	\SIA=&\frac{i}{2}\int\dd^4x\dd^4x^\prime\varphi_{\bar{X}I}^\Delta(x)\Re[\Pi_{XY}{}^{IJ}(x,x^\prime)]\varphi_{\bar{Y}J}^{\Delta}(x^\prime) \nonumber \\
	&-2\epsilon\int\dd^4x\dd^4x^\prime\theta(t-t^\prime)\varphi_{\bar{X}I}^\Delta(x)\Im[\Pi_{XY}{}^{IJ}(x,x^\prime)]\bar{\varphi}_{\bar{Y}J}(x^\prime),
}
where
\bae{
	\Pi_{XY}{}^{IJ}(x,x^\prime)&=\int\dk\ee^{i\mathbf{k}\cdot(\mathbf{x}-\mathbf{x}^\prime)}\delta(t-t_\sigma)\delta(t^\prime-t_\sigma)
	Q_{X\alpha}^I(t,k)Q^{*J}_{Y\alpha}(t^\prime,k) \nonumber \\
	&=\frac{\dot{k}_\sigma}{k_\sigma}\frac{\sin k_\sigma r}{k_\sigma r}\calP_{XY}{}^{IJ}(k_\sigma)\delta(t-t^\prime),
}
with
\bae{\label{eq: calP}
	\calP_{XY}{}^{IJ}(k)=\frac{k^3}{2\pi^2}Q_{X\alpha}^I(k)Q_{Y\alpha}^{*J}(k).
}
The first term yields the noise term as we will see and the second term gives the hateful dissipation and mass renormalization which should be
removed.


\subsection{Langevin equation}

Hereafter we basically assume $\epsilon=0$.
Let us first discuss the meaning of the Keldysh basis. According to the classicality assumption, we are interested in the dynamics of
the vacuum expectation value $\braket{\varphi^+}=\braket{\varphi^-}$, which is given by $\braket{\bar{\varphi}}$ in the Keldysh basis.
Then the other mode $\varphi^\Delta$ represents the deviation from this trajectory.
Therefore the EoM for $\bar{\varphi}$ can be obtained by the variation of the effective action in $\varphi^\Delta$ at the condition $\varphi^\Delta=0$.
However the influence action~(\ref{eq: SIA}) is quadratic in $\varphi^\Delta$, and hence it does not contribute to the EoM.
Accordingly the variation of the effective action~(\ref{eq: GammaIR}) is merely equal to the classical EoM~(\ref{eq: classical IR EoM}).
It can be interpreted as the fact that taking average of the Langevin equation reduces it to the classical EoM since the mean value of noise is 
zero.\footnote{Strictly speaking, if the non-linearity of $\varphi$ is significant in the EoM, the simple average of the Langevin equation itself is not
reduced to the classical EoM as an important feature of the stochastic formalism. In this case, deriving EoM directly from 
the complex effective action~(\ref{eq: GammaIR}) leads to a wrong result.}

The key point is that the influence action is quadratic and pure imaginary. Therefore the probability weight in the partition function
$\ee^{i\SIA}$ would be Gaussian, suggesting the Gaussian auxiliary field introduced by its functional Fourier transformation~\cite{Caldeira:1982iu,Grabert:1988yt}
\bae{\label{eq: functional Fourier trs}
	\ee^{i\SIA}=\calN\int\scrD\xi_X^IP[\xi_X^I]\ee^{i\int\dd^4x\varphi_{\bar{X}I}^\Delta\xi_X^I},
}
where
\bae{
	P[\xi_X^I]=\calN\exp\left[-\frac{1}{2}\int\dd^4x\dd^4x^\prime\xi_{X}^I(x)(\Re\Pi^{-1})_{XYIJ}(x,x^\prime)\xi_Y^J(x^\prime)\right].
}
The Gaussian weight $P[\xi_X^I]$ determines the statistical properties of $\xi$ as
\bae{
	\bce{
		\dps
		\braket{\xi_X^I(x)}=\int\scrD\xi_X^I\,\xi_X^I(x)P[\xi_X^I]=0, \\[10pt]
		\dps
		\braket{\xi_X^I(x)\xi_{Y}^J(x^\prime)}=\int\scrD\xi_X^I\,\xi_X^I(x)\xi_{Y}^J(x^\prime)P[\xi_X^I]=\Re\Pi_{XY}{}^{IJ}(x,x^\prime),
	}
}
and the other part in Eq.~(\ref{eq: functional Fourier trs}) can be interpreted as a new coupling between $\varphi$ and $\xi$ 
in the effective action.
After this transformation, the effective action reads pure real as
\bae{\label{eq: Seff}
	S_\eff=S^{(0)+}-S^{(0)-}+\int\dd^4x\varphi_{\bar{X}I}^\Delta\xi_X^I.
}
%\bae{\label{eq: Seff}
%	S_\eff=S^{(0)+}-S^{(0)-}+\int\dd^4x\varphi_{\bar{X}I}^\Delta\xi_X^I
%	-2\epsilon\int\dd^4x\dd^4x^\prime\theta(t-t^\prime)\varphi_{\bar{X}I}^\Delta(x)\Im[\Pi_{XY}{}^I{}_J(x,x^\prime)]\bar{\varphi}_{\bar{Y}}^{J}(x^\prime)
%}
Finally the variation of this real effective action in $\varphi^\Delta$ at $\varphi^\Delta=0$ gives a set of Langevin equations, which can be covariantized as
\bae{\label{eq: covariant Langevin}
	&\left.\var{S_\eff}{\varpi_I^\Delta}\right|_{\varphi^\Delta=0}=0, \quad\quad \left.\var{S_\eff}{\varphi^{I\Delta}}\right|_{\varphi^\Delta=0}=0, \nonumber \\
    &\Leftrightarrow \quad \dot{\bar{\varphi}}^I=\frac{N_\IR}{a^3}G^{IJ}\bar{\varpi}_J+\xi_Q^I, \quad\quad D_t\bar{\varpi}_I=-a^3N_\IR V_I+\xi_{\tilde{P}I},
}
consistently with the heuristic approach~(\ref{eq: Langevin in heuristic})

If one includes the second term of the influence action~(\ref{eq: SIA}), it yields additional dissipation and mass renormalization terms proportional to
$\bar{\varphi}$ and $\bar{\varpi}$, which is inconsistent with the original theory even in the test particle limit.
That is why one must get rid of the latter term in the IR-UV coupling~(\ref{eq: IR-UV in Keldysh}).
Also note that the influence action does not depends on $N_\IR$ explicitly. 
Therefore the Friedmann constraint corresponding with the variation in $N_\IR$
\bae{\label{eq: Friedmann eq}
	\left.\var{S_\eff}{N_\IR^\Delta}\right|_{\varphi^\Delta=0}=0, \quad \Leftrightarrow \quad 
	\frac{3\Mpl^2\calH^2}{N_\IR^2}=\frac{1}{2a^6}G^{IJ}\bar{\varpi}_I\bar{\varpi}_J+V,
}
still holds in each Hubble patch in the stochastic formalism as we saw in Eq.~(\ref{eq: perturbation energy const}).
In the rest of this paper, we will omit bars of $\bar{\varphi}$ and $\bar{\varpi}$ for conciseness.


%%%%%%%%%%%%%%%%%%%%%%%%%%%%%%%%%%%
\begin{comment}
Let us check the consistency of the obtained Langevin equation in the free test fields case. That is, here we assume $V\simeq\text{const.}$,
$G^{IJ}\varpi_I\varpi_J/2a^6\ll V$, and $G_{IJ}=\delta_{IJ}$. In this case, the solution for the mode function is given by
\bae{
	Q^I_\alpha=\frac{1}{\sqrt{2k}a}\left(1-\frac{i}{k\eta}\right)\ee^{-ik\eta}\delta^I_\alpha.
}
By adopting the e-folding number $N$ for a time variable, i.e. $N_\IR=1/H$, the Langevin equation can be written as
\bae{
	\dif{\varphi^I}{N}=\tilde{\varpi}_I+\xi_Q^I, \quad\quad
	\dif{\tilde{\varpi}_I}{N}=-3H\tilde{\varpi}_I+\tilde{\xi}_{PI},
}
where $\tilde{\varpi}_I=\varpi_I/a^3H$ and $\tilde{\xi}_{PI}=\xi_{PI}/a^3H$.
The noise amplitude is 
\bae{
	\braket{\xi_Q^I(x)\xi_Q^J(x^\prime)}=\left(\frac{H}{2\pi}\right)^2\delta^{IJ}\delta(N-N^\prime)\frac{\sin k_\sigma r}{k_\sigma r}
	+\calO(\sigma^2),
}
and all the other combinations are suppressed as $\calO(\sigma^2)$.
Therefore the system is rapidly settled down to the attractor trajectory as $\tilde{\varpi}_I\to0$
and the inflaton field values are given by simple Brownian motions:
\bae{
	\varphi^I(N,\mathbf{x})=\int^N_{N_0}\xi_Q^I(N^\prime,\mathbf{x})\dd N^\prime,
}
where $N_0$ is some initial time.
\end{comment}
%%%%%%%%%%%%%%%%%%%%%%%%%%%%%%%%%%%



%%%%%%%%%%%%%%%%%%%%%%%%%%%%%%%%%%%%
\begin{comment}

\section{Simple example}\label{sec: example}

\subsection{Free test particle}

Let us check the consistency of the obtained Langevin equation in the free test fields case. That is, here we assume $V\simeq\text{const.}$,
$G^{IJ}\varpi_I\varpi_J/2a^6\ll V$, and $G_{IJ}=\delta_{IJ}$. In this case, the solution for the mode function is given by
\bae{
	Q^I_\alpha=\frac{1}{\sqrt{2k}a}\left(1-\frac{i}{k\eta}\right)\ee^{-ik\eta}\delta^I_\alpha.
}
By adopting the e-folding number $N$ for a time variable, i.e. $N_\IR=1/H$, the Langevin equation can be written as
\bae{
	\dif{\varphi^I}{N}=\tilde{\varpi}_I+\xi_Q^I, \quad\quad
	\dif{\tilde{\varpi}_I}{N}=-3H\tilde{\varpi}_I+\tilde{\xi}_{PI},
}
where $\tilde{\varpi}_I=\varpi_I/a^3H$ and $\tilde{\xi}_{PI}=\xi_{PI}/a^3H$.
The noise amplitude is 
\bae{
	\braket{\xi_Q^I(x)\xi_Q^J(x^\prime)}=\left(\frac{H}{2\pi}\right)^2\delta^{IJ}\delta(N-N^\prime)\frac{\sin k_\sigma r}{k_\sigma r}
	+\calO(\sigma^2),
}
and all the other combinations are suppressed as $\calO(\sigma^2)$.
Therefore the system is rapidly settled down to the attractor trajectory as $\tilde{\varpi}_I\to0$
and the inflaton field values are given by simple Brownian motions:
\bae{
	\varphi^I(N,\mathbf{x})=\int^N_{N_0}\xi_Q^I(N^\prime,\mathbf{x})\dd N^\prime,
}
where $N_0$ is some initial time.


\subsection{Cartesian vs. polar coordinate}

As an exercise of the coordinate transformation, let us study the canonical two-field model:
\bae{
	S=\int\dd^4x\sqrt{-g}\left[\frac{1}{2}\Mpl^2\calR-\frac{1}{2}\partial_\mu X\partial^\mu X-\frac{1}{2}\partial_\mu Y\partial^\mu Y
	-\frac{1}{2}M^2X^2-\frac{1}{2}m^2Y^2\right],
}
in the polar coordinate as
\bae{
	\phi^r=r=\sqrt{X^2+Y^2}, \quad\quad \phi^\theta=\theta=\tan^{-1}\frac{Y}{X}.
}
In this coordinate the field space metric is given by
\bae{
	G_{rr}=1, \quad\quad G_{\theta\theta}=r^2, \quad\quad (\text{otherwise})=0.
}
The corresponding affine connection is
\bae{
	\Gamma^r_{\theta\theta}=-r, \quad\quad \Gamma^\theta_{r\theta}=\frac{1}{r}, \quad\quad (\text{otherwise})=0.
}
The Riemann tensor should vanish since the original field space is flat.

\end{comment}

%%%%%%%%%%%%%%%%%%%%%%%%%%%%%%%



\section{Complementaries}\label{sec: complementaries}

Even though one has a Langevin equation itself, it is not enough in order to implement the stochastic formalism practically.
First the noise amplitude $\calP_{XY}{}^{IJ}(k_\sigma)$ should be obtained.
It is determined by the subhorizon dynamics with an appropriate initial condition.
We also mention the uncertainty on the mathematical prescription of noise integral.


\subsection{Subhorizon equation of motion}\label{sec: UV EoM}

While we focused on the dynamics of IR sectors, one also has to consider the UV modes to calculate the noise amplitude $\calP_{XY}{}^{IJ}$.
In this section we discuss the linearized EoM for subhorizon modes as well as their initial condition. 
First, by substituting the momentum constraint~(\ref{eq: alpha}) into Eq.~(\ref{eq: S2 cov}), 
the quadratic action $S^{(2)}$ can be summarized in the matrix form as
\bae{
	S^{(2)}&=\int\dd^4x\frac{1}{2}Q_{X}^I\Lambda_{XYIJ}Q_Y^J, \nonumber \\
	\bpme{
		\Lambda_{QQIJ}, & \Lambda_{QPI}{}^J \\
		\Lambda_{PQ}{}^I{}_J, & \Lambda_{PP}{}^{IJ}
	}&=\bpme{
		a^3N_\IR\left(G_{IJ}\frac{\nabla^2}{a^2}-M_{QQIJ}^2\right), & -\delta^J_ID_t-M_{PQ}^2{}^J{}_I \\
		\delta^I_JD_t-M_{PQ}^2{}^I{}_J, & -\frac{N_\IR}{a^3}G^{IJ}
	},
}
where
\bae{
	M^2_{QQIJ}=&V_{IJ}+\frac{N_\IR^2}{2\Mpl^2\calH a^9}G^{KL}{}_{,I}\varpi_K\varpi_L\varpi_J+\frac{1}{2a^6}G^{KL}{}_{,IJ}\varpi_K\varpi_L \nonumber \\
    &+\frac{N_\IR}{2\Mpl^2\calH a^3}(V_I\varpi_J+V_J\varpi_I)+\frac{3}{2\Mpl^2a^6}\varpi_I\varpi_J, \\
	M^2_{PQ}{}^I{}_J=&\frac{N_\IR^2}{2\Mpl^2\calH a^6}G^{IK}\varpi_K\varpi_J+\frac{N_\IR}{a^3}G^{IK}{}_{,J}\varpi_K.
}
As described in Sec.~\ref{sec: influence action}, the inverse of this $\Lambda$ operator indeed represents the propagator of the subhorizon modes.
Therefore the EoM for UV modes are given by
\bae{\label{eq: UV EoM}
	\Lambda_{XYIJ}Q_Y^J=0, \quad\quad \Leftrightarrow \quad\quad 
	\bce{
		\dps
		D_tQ^I\approx\frac{N_\IR}{a^3}\tilde{P}^I+\tilde{M}^2_{PQ}{}^I{}_JQ^J, \\[10pt]
		\dps
		D_t\tilde{P}_I\approx a^3N_\IR\left(G_{IJ}\frac{\nabla^2}{a^2}-\tilde{M}_{QQIJ}^2\right)Q^J-\tilde{M}^2_{PQI}{}^J\tilde{P}_J,
	}
}
with
\bae{
	\tilde{M}^2_{QQ}{}^I{}_J&=V^I{}_{;J}-\frac{1}{a^6}R^{IKL}{}_J\varpi_K\varpi_L+\frac{N_\IR}{2\Mpl^2\calH a^3}(V^I\varpi_J+V_J\varpi^I)+\frac{3}{2\Mpl^2a^6}\varpi^I\varpi_J, \\
    \tilde{M}^2_{PQ}{}^I{}_J&=\frac{N_\IR^2}{2\Mpl^2\calH a^6}\varpi^I\varpi_J.
}
Here we used the classical EoM~(\ref{eq: classical IR EoM}) for the right side expressions in Eq.~(\ref{eq: UV EoM}).
It can be reduced to the well-known formula~\cite{Sasaki:1995aw},
\bae{\label{eq: 2nd order UV EoM}
	&\frac{1}{N_\IR^2}D_t^2Q^I+\frac{1}{N_\IR^2}\left(3\calH-\frac{\dot{N}_\IR}{N_\IR}\right)D_tQ^I \nonumber \\
	&+\left[-\delta^I_J\frac{\nabla^2}{a^2}+V^I{}_{;J}-\frac{1}{a^6}R^{IKL}{}_J\varpi_K\varpi_L
	-\frac{1}{a^3N_\IR\Mpl^2}D_t\left(\frac{N_\IR}{a^3\calH}\varpi^I\varpi_J\right)\right]Q^J\approx0,
}
with use of the time-derivative of the Friedmann equation
\bae{ 
	\frac{\dot{\calH}}{\calH}-\frac{\dot{N}_\IR}{N_\IR}\approx-\frac{N_\IR^2}{2\Mpl^2\calH a^6}\varpi_I\varpi^I.
}
Again note that the approximated equality ($\approx$) only holds up to noise. However the noise term itself comes from the UV modes
and therefore can be neglected for the linearized UV EoM.
One can rewrite these equations into the form exhibited in Eq.~(\ref{eq: decomposed EoM})
\bae{
	E_Q^I(t,\mathbf{k})=&-\dot{\phi}^I(t,\mathbf{k})+\frac{N_\IR}{a^3}G^{IJ}\pi_J(t,\mathbf{k})
	+\frac{\alpha(t,\mathbf{k})}{a^3}G^{IJ}\varpi_J+\frac{N_\IR}{a^3}G^{IJ}{}_{,K}\varpi_J\phi^K(t,\mathbf{k})=0, \\
	E_{PI}(t,\mathbf{k})=&-\dot{\pi}_I(t,\mathbf{k})-a^3N_\IR\phi^J(t,\mathbf{k})\partial_JV_I-a^3\alpha(t,\mathbf{k})V_I
	-aN_\IR G_{IJ}k^2\phi^J(t,\mathbf{k})-\frac{\varpi_Ik^2}{a^2}\psi(t,\mathbf{k})
	\nonumber \\
	&-\frac{\alpha(t,\mathbf{k})}{2a^3}G^{JK}{}_{,I}\varpi_J\varpi_K 
	-\frac{N_\IR}{a^3}G^{JK}{}_{,I}\varpi_J\pi_K(t,\mathbf{k})-\frac{N_\IR}{2a^3}G^{JK}{}_{,IL}\varpi_J\varpi_K\phi^L(t,\mathbf{k})=0,
}
with use of the energy and momentum constraints~(\ref{eq: perturbation energy const}) and (\ref{eq: alpha})
\bae{
	\bce{
		\dps
		\frac{\nabla^2}{a^2}\psi=-\frac{3\calH}{N_\IR}\alpha
		-\frac{N_\IR^2}{2\Mpl^2\calH}\left(\frac{1}{a^6}G^{IJ}\varpi_IP_J+\frac{1}{2a^6}G^{IJ}{}_{,K}\varpi_I\varpi_JQ^K+V_IQ^I\right), \\[10pt]
		\dps
		\alpha=\frac{N_\IR^2}{2\Mpl^2\calH a^3}\varpi_IQ^I.
	}
}

Here let us mention the effect of the metric perturbations.
While metric perturbations do not affect the Langevin equation itself as we saw,
the UV modes are affected by them in two ways. One is that they give rise to a new mixing term
$-\frac{1}{a^3N_\IR\Mpl^2}D_t\left(\frac{N_\IR}{a^3\calH}\varpi^I\varpi_J\right)Q^J$ in EoM~(\ref{eq: 2nd order UV EoM}).
The other one is that the momentum conjugate $\tilde{P}_I$ which directly causes the noise for $\varpi_I$ is not simply given by
the covariant time derivative $\frac{a^3}{N_\IR}D_tQ_I$ but mixed with $Q^I$ by $\frac{a^3}{N_\IR}\tilde{M}^2_{PQIJ}Q^J$.

We now derived the linearized UV EoM.
To obtain the noise amplitude, one has to solve these EoM with an appropriate initial condition.
First the mode function is defined by Eq.~(\ref{eq: mode expansion})
\bae{
	\bce{
		\dps
		\hat{Q}^I(x)=\int\dk\left[Q^I_\alpha(t,k)\hat{a}_\alpha(\mathbf{k})\ee^{i\kdx}
		+Q^{*I}_\alpha(t,k)\hat{a}^\dagger_\alpha(\mathbf{k})\ee^{-i\kdx}\right], \\[10pt]
		\dps
		\hat{\tilde{P}}_I(x)=\int\dk\left[\tilde{P}_{I\alpha}(t,k)\hat{a}_\alpha(\mathbf{k})\ee^{i\kdx}
		+\tilde{P}^*_{I\alpha}(t,k)\hat{a}^\dagger_\alpha(\mathbf{k})\ee^{-i\kdx}\right],
	}
}
with the creation-annihilation operator normalized by Eq.~(\ref{eq: aadagger})
\bae{\label{eq: aadagger App}
	[\hat{a}_\alpha(\mathbf{k}),\hat{a}^\dagger_\beta(\mathbf{q})]=(2\pi)^3\delta_{\alpha\beta}(\mathbf{k}-\mathbf{q}).
}
In the subhorizon limit, one can find the $\frac{k}{a}$-leading solution of Eq.~(\ref{eq: UV EoM}) by the WKB form up to the overall normalization as
\bae{
	\bce{
		\dps
		Q^I_\alpha(t,k)=f(t,k)e^I_\alpha(t)\exp\left[-i\int\frac{kN_\IR}{a}\dd t\right], \\[10pt]
		\dps
		\tilde{P}_{I\alpha}(t,k)=-ia^2kf(t,k)e_{I\alpha}(t)\exp\left[-i\int\frac{kN_\IR}{a}\dd t\right],
	}
}
where $e^I_\alpha$ is the parallel transported vielbein:\footnote{Once the vielbein is determined by the normalization 
condition~(\ref{eq: normalization of vielbein}) at some initial time, the solution of the parallel transport condition~(\ref{eq: parallel transport condition})
always satisfies its normalization~\cite{Weinberg:2008zzc} irrespectively of the stochasticity of $\dot{\varphi}$.}
\bae{
	&e^I_\alpha(t) e^J_\alpha(t)=G^{IJ}(\varphi(t)), \label{eq: normalization of vielbein} \\
	&D_te^I_\alpha=\dot{e}^I_\alpha+\Gamma^I_{JK}\dot{\varphi}^Je^K_\alpha=0. \label{eq: parallel transport condition}
}
By imposing the canonical quantization condition as\footnote{Note that the canonical quantization is defined by the commutator of $Q$ and $D_tQ$
in the literature~\cite{Dias:2015rca,Ronayne:2017qzn}. But the difference between $\tilde{P}$ and $D_tQ$ is proportional to $Q$
which does not affect the commutator with $Q$ itself, and therefore our quantization is equivalent.}
\bae{
	[\hat{Q}^I(t,\mathbf{x}),\hat{\tilde{P}}_J(t,\mathbf{y})]=i\delta^I_J\delta^{(3)}(\mathbf{x}-\mathbf{y}),
}
and with use of the commutation relation~(\ref{eq: aadagger App}), the normalization of $f(t,k)$ is determined as
\bae{
	|f(t,k)|^2=\frac{1}{2ka^2}.
}
Therefore, up to the irrelevant constant phase shift, the initial condition of the mode functions is given by
\bae{
	\bce{
		\dps
		Q^I_\alpha(t,k)=\frac{1}{\sqrt{2k}a}e^I_\alpha\exp\left[-i\int\frac{kN_\IR}{a}\dd t\right], \\[10pt]
		\dps
		\tilde{P}_{I\alpha}(t,k)=-i\sqrt{\frac{k}{2}}ae_{I\alpha}\exp\left[-i\int\frac{kN_\IR}{a}\dd t\right].
	}
}
By solving EoM~(\ref{eq: UV EoM}) analytically or numerically with this initial condition, one can calculate the noise correlation~(\ref{eq: calP})
\bae{
	\calP_{XY}{}^{IJ}(k_\sigma)=\frac{k^3}{2\pi^2}Q_{X\alpha}^I(k_\sigma)Q_{Y\alpha}^{*J}(k_\sigma).
}


\subsubsection{Neglecting mixing}

As a simple exercise, let us consider the case where the all mixing terms can be neglected and the EoM for the mode function is approximated as
\bae{
	D_tQ^I_\alpha\simeq\frac{N_\IR}{a^3}\tilde{P}^I_\alpha, \quad\quad 
	D_t\tilde{P}_{I\alpha}\simeq-a^3N_\IR\left(\frac{k^2}{a^2}+m_\alpha^2(t)\right)Q_{I\alpha}.
}
Here the summation in the index $\alpha$ is not taken and $m_\alpha^2(t)$ represents a time-depending effective mass squared 
for $\alpha$ mode. With use of the parallel transported vielbein $e^I_\alpha$~(\ref{eq: parallel transport condition}), 
the solution is formally given by
\bae{
	Q^I_\alpha(t,k)=g_\alpha(t,k)e^I_\alpha, \quad\quad \tilde{P}_{I\alpha}(t,k)=\frac{a^3}{N_\IR}\dot{g}_\alpha(t,k)e_{I\alpha},
}
where $g(t,k)$ satisfies single-field like EoM
\bae{
	\frac{1}{N_\IR^2}\ddot{g}_\alpha(t,k)+\frac{1}{N_\IR^2}\left(3\calH-\frac{\dot{N}_\IR}{N_\IR}\right)\dot{g}_\alpha(t,k)
	+\left(\frac{k^2}{a^2}+m^2_\alpha(t)\right)g_\alpha(t,k)=0.
}
Again the $\alpha$-summation is not taken. Then the noise amplitude is given by
\bae{
	\bce{
		\dps
		\calP_{QQ}{}^{IJ}(k_\sigma)=\frac{k^3}{2\pi^2}\sum_\alpha g_\alpha(k_\sigma)g^*_\alpha(k_\sigma)e^I_\alpha e^J_\alpha, \\[10pt]
		\dps
		\calP_{QP}{}^I{}_J(k_\sigma)=\calP^*_{PQJ}{}^I=\frac{k^3}{2\pi^2}\frac{a^3}{N_\IR}\sum_\alpha g_\alpha(k_\sigma)\dot{g}^*_\alpha(k_\sigma)e^I_\alpha e_{J\alpha}, \\[10pt]
		\dps
		\calP_{PPIJ}(k_\sigma)=\frac{k^3}{2\pi^2}\frac{a^6}{N_\IR^2}\sum_\alpha\dot{g}_\alpha(k_\sigma)\dot{g}^*_\alpha(k_\sigma)e_{I\alpha}e_{J\alpha}.
	}
}

In the constant-mass and de Sitter approximation, the solution for $g_\alpha$ reads a well-known Hankel-function type:
\bae{
	g_\alpha(t,k)=\frac{1}{a}\sqrt{\frac{\pi}{4k}}\sqrt{x}H_{\nu_\alpha}^{(1)}(x), \quad\quad 
	\left(x=-k\tau \text{ and } \nu_\alpha=\sqrt{\frac{9}{4}-\frac{m_\alpha^2}{H^2}}\right),
}
where $\tau$ is a conformal time which is equivalent to $-1/aH$ in the de Sitter limit and $H_\nu^{(1)}$ is the Hankel function of the first kind.
Moreover, if all the masses are sufficiently small as $\nu_\alpha\to3/2$, the noise amplitude is much simplified as
\bae{
	\calP_{QQ}{}^{IJ}=\left(\frac{H}{2\pi}\right)^2(1+\sigma^2)G^{IJ}, \quad\quad \calP_{QP}{}^I{}_J=\frac{a^3H^3}{4\pi^2}(-\sigma^2+i\sigma^3)\delta^I_J,
	\quad\quad \calP_{PPIJ}=\frac{a^6H^4}{4\pi^2}\sigma^4G_{IJ},
}
for the cosmic time $N_\IR=1$.
Therefore almost all terms are suppressed by $\sigma$ except for the standard noise amplitude $(H/2\pi)^2$ in $\calP_{QQ}$.
All of the dangerous imaginary parts of $\calP$ are also higher order in $\sigma$.



\subsection{It\^o vs. Stratonovich}

We have seen that the noise for IR modes has a white spectrum as a consequence of the sharp step window function.
Though white noise is easily to handle with, it should be interpreted as a limit of a realistic colored noise corresponding with a physical smooth window.
At the price of this simplification, there is an uncertainty on the mathematical prescription of noise integral. We here briefly mention this problem.

Equations of motion with random noise are mathematically dealt with by stochastic calculus.
In stochastic calculus, the definition of the noise integral has an uncertainty whether 
the noise amplitude is evaluated before or after the noise kick.
Representatively, the It\^o integral evaluates the noise amplitude just before the noise kick, while the evaluation point is a middle point between before and after the noise kick in the Stratonovich type definition.
In general, the It\^o's definition is preferred to reflect the fact that nobody knows the future, and in fact many authors also have interpreted the Langevin equations in the It\^o's way from the viewpoint of causality in the context of stochastic inflation~\cite{Salopek:1990re,Vilenkin:1999kd,Tolley:2008na,Fujita:2014tja,Vennin:2015hra,Tokuda:2017fdh}.
However in physics, the Stratonovich integral also appears in some cases because the Langevin equation is often derived in some limits.
Ref.~\cite{Mezhlumian:1991hw} claims that the noise amplitude should be evaluated at the middle point of the window as the Stratonovich integral
since the sharp window is a limit of the physical smooth one.

We point out that there are drawbacks both for the It\^o and Stratonovich interpretations.
For the It\^o integral, it is known that the ordinary differential calculus does not hold but the chain rule is corrected by the noise amplitude.
If variables $X^I$ satisfy the Langevin equation
\bae{
	\dot{X}^I=f^I+\xi^I,
}
the equation of motion for redefined fields $\tilde{X}^{\tilde{I}}$ is given by the It\^o's lemma as
\bae{
	\dot{\tilde{X}}^{\tilde{I}}=\pdif{\tilde{X}^{\tilde{I}}}{X^J}\left(f^J+\xi^J\right)+\frac{1}{2}\frac{\partial^2\tilde{X}^{\tilde{I}}}{\partial X^J\partial X^K}A^{JK},
}
where $A^{JK}$ represents the noise amplitude: $\braket{\xi^I(t)\xi^J(t^\prime)}=A^{JK}\delta(t-t^\prime)$. Due to the last term, the Langevin equation is not
actually covariant under the field redefinition if it is interpreted in the It\^o's way.
On the other hand, if one interprets the Langevin equation as a Stratonovich type,
the field space invariance is manifest because the ordinary chain rule does hold for the Stratonovich SDE.
In this case, however, the Stratonovich process depends on the choice of the square root of the noise amplitude, which is not unique in a multi-field case since
the noise amplitude is a matrix. The subhorizon dynamics only tells us the noise amplitude of the correlation function, and therefore the Stratonovich-type Langevin
equation is undetermined in principle. 
%Note that both drawbacks are proportional to the noise correlator (i.e. quadratic order of the noise) and there is no difference between the It\^o and Stratonovich definition at the linear order in noise.
We will address these issues in separated articles~\cite{1st,3rd,4th}.


%%%%%%%%%%%%%%%%%%%%%%%%%
\begin{comment}
Under the field space transformation $\varphi^I\to\varphi^{\prime A}(\varphi^I)$, the drift term $\frac{N_\IR}{a^3}\varpi^I$ for $\dot{\varphi}^I$ does not transform as a standard vector $\frac{N_\IR}{a^3}\varpi^{\prime A}=\pdif{\varphi^{\prime A}}{\varphi^I}\frac{N_\IR}{a^3}\varpi^I$ but does as $\frac{N_\IR}{a^3}\varpi^{\prime A}=\pdif{\varphi^{\prime A}}{\varphi^I}\frac{N_\IR}{a^3}\varpi^I+\frac{1}{2}\frac{\partial^2\varphi^{\prime A}}{\partial\varphi^I\partial\varphi^J}\Re\Pi_{QQ}{}^{IJ}$ according to the It\^o's lemma.
It means that the Hubble parameter given by the Friedmann equation~(\ref{eq: Friedmann eq}) is no longer field space invariant though it should be.
On the other hand, if one interprets the Langevin equation as a Stratonovich type,
the field space invariance is manifest because the ordinary chain rule does hold for the Stratonovich SDE.
In this case, however, the Stratonovich process is not invariant under the rotation of the noise basis and therefore it is undetermined because we only know the correlation of the noise.
Note that both drawbacks are proportional to the noise correlator (i.e. quadratic order of the noise) and there is no difference between the It\^o and Stratonovich definition at the linear order in noise.
We will address these issues in a separated article.
\end{comment}
%%%%%%%%%%%%%%%%%%%%%%%%%%








\section{Conclusions}\label{sec: conclusions}

The stochastic formalism is formulated in a curved field space. The effective Hamiltonian action approach is used as well as the heuristic
equation of motion method, including metric perturbations.
Though they do not affect the form of EoM for coarse-grained (IR) modes, metric perturbations alter the subhorizon (UV) dynamics.
Also the explicit appearance of the lapse function reveals that the Friedmann equation still holds in each Hubble patch without any noise.
In the last section, we briefly mention the uncertainty of noise integral represented by the It\^o or Stratonovich process, which will be addressed
in separated papers~\cite{1st,3rd,4th}.

As recently pointed out by Tokuda and Tanaka~\cite{Tokuda:2017fdh}, the IR-UV coupling is chosen ad hoc so that the two point function is reproduced 
at the free particle limit. More rigorous understanding of this problem has not been reached and left for future works.
The effective action approach in the path integral method can be naturally extended to the loop calculations beyond the tree level.
In this case, the time-independent counter terms should be properly taken so that only the UV loop contributions are cancelled~\cite{Tokuda:2017fdh}.
We will also address this topic in a coming article~\cite{4th}.

One of the characteristic phenomena of curved field spaces is the so-called geometrical destabilization of inflation~\cite{Renaux-Petel:2015mga,
Renaux-Petel:2017dia,Garcia-Saenz:2018ifx}. The negative field space curvature gives an effective destabilizing force for entropic directions~\cite{Renaux-Petel:2015mga}.
Therefore, if this geometrical destabilizing force grows during inflation, the entropic directions can suddenly develop large perturbations dominating 
the homogeneous mode even if those directions are stabilized originally. Compared to the classical hybrid inflation model, the remarkable point of this mechanism
is that the second slow-roll phase would take place after the destabilization point~\cite{Garcia-Saenz:2018ifx} and therefore possibly affects not only on
invisibly small scales but also on observable large scales.
Since the perturbative expansion is broken down around the critical point due to large fluctuations, one must resort to a non-perturbative method represented e.g. by
the stochastic formalism which we derived here, and we left it for future works.




%%%%%%%%%%%%%%%%%%%%%%%%%%%
\begin{comment}
Inspired by recent progress in the framework of curved-field-space inflation like 
\emph{geometrical destabilization}~\cite{Renaux-Petel:2015mga,Renaux-Petel:2017dia}, \emph{hyperinflation}~\cite{Brown:2017osf,Mizuno:2017idt},
and so on, we formulate the stochastic inflation in the curved field space case in this paper.
The fully covariant Langevin equations~(\ref{eq: covariant Langevin}) are derived in the effective Hamiltonian action approach, 
including metric perturbations.
While the metric perturbations only affect the equation of motion of the subhorizon mode functions,
the explicit appearance of the lapse function reveals that the Friedmann equation~(\ref{eq: Friedmann eq}) still holds in each Hubble patch 
without any noise.

As recently pointed out by Tokuda and Tanaka~\cite{Tokuda:2017fdh}, there is a fundamental problem in the time-depending decomposition
of the path integral measure and it is not completely understood but left as an important subject of future works.
Also we will study the behavior around or after the critical point in the geometrical destabilization framework with use of the obtained Langevin equations.
\end{comment}
%%%%%%%%%%%%%%%%%%%%%%%%%%%%





\acknowledgments

We would like to thank Takahiro Tanaka, Junsei Tokuda and Vincent Vennin for fruitful discussions.
YT is supported by grants from R\'egion \^Ile-de-France and Grand-in-Aid for JSPS Research Fellow (JP18J01992).




%\appendix






\bibliographystyle{JHEP}
\bibliography{non-flat-stochastic}
\end{document}


















